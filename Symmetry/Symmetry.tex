\documentclass{article}
\usepackage[top=25truemm,bottom=20truemm,left=20truemm,right=20truemm]{geometry}

\usepackage{amsmath}
\usepackage{amssymb}
\usepackage{amsfonts}
\usepackage{mathrsfs}
\usepackage{latexsym}
\usepackage{bm}
\usepackage[dvipdfmx]{graphicx}
\usepackage{physics}
\usepackage{braket}
\usepackage{float}

\usepackage{comment}

\title{Notes on symmetry and group theory}
\author{Ryota Masuki}
\date{\today}

\begin{document}
\maketitle

\section{Time-reversal and the representations of non-unitary group}
This section is based on the discussion in the textbook by Inui, et al~\cite{inui2012group}.

\subsection{Time-reversal operation}
If we define the time-reversal operator as
\begin{align}
  \theta = - i \sigma_y K = 
  \begin{pmatrix}
     & -1 \\
    1 &  \\
  \end{pmatrix}
  K,
\end{align}
where $K$ is the complex conjugate operator.

\subsection{Non-unitary group and corepresentations}
When the unitary operations $u_i, u_j, u_k$ satisfy $u_i u_j = u_k$, 
\begin{align}
  (\theta u_i) u_j = u_i (\theta u_j) = \theta u_k
\end{align}
\begin{align}
  (\theta u_i) (\theta u_j) = (-1)^N u_k,
\end{align}
where $N$ is the number of electrons. The magnetic group $\bar{G}$ can be written as
\begin{align}
  \bar{G} = G + \theta G,
\end{align}
where $G$ is the set of unitary symmetry operations (symmetry operations without time-reversal).

The corepresentations of the non-unitary group satisfy
\begin{align}
  u_1 u_2 \psi_k &= \sum_j \psi_j [D(u_1) D(u_2)]_{jk}
  \label{eq:corepresentation_u1u2}
  \\
  u a \psi_k &= \sum_j \psi_j [D(u) D(a)]_{jk}
  \label{eq:corepresentation_ua}
  \\
  au \psi_k &= \sum_j \psi_j [D(a) D^*(u)]_{jk}
  \label{eq:corepresentation_au}
  \\
  a_1 a_2 \psi_k &= \sum_j \psi_j [D(a_1) D^*(a_2)]_{jk},
  \label{eq:corepresentation_a1a2}
\end{align}
where $u, u_1, u_2$ are unitary operations and $a, a_1, a_2$ are anti-unitary operations in the group (We use this notation without explanation from here on).
Two corepresentations $D$ and $D'$ are defined to be equivalent if there is a matrix $T$ that satisfy
\begin{align}
  D'(u) &= T^{-1} D(u) T
  \label{eq:equiv_corep1}
  \\
  D'(a) &= T^{-1} D(a) T^*
  \label{eq:equiv_corep2}
\end{align}
Note that the relation Eqs. (\ref{eq:corepresentation_u1u2}) to (\ref{eq:corepresentation_a1a2}) is unchanged by the transformation Eq. (\ref{eq:equiv_corep1}) and (\ref{eq:equiv_corep2}).

\subsection{Construction of the corepresentations}
Here, we consider how to construct the corepresentations of $\bar{G}$ from the representations of its unitary part $G$.
We first start from an irreducible representation and a basis of $G$
\begin{align}
  u \psi_k = \sum_{j = 1}^d \Delta_{jk}(u).
\end{align}
Here, we define 
\begin{align}
  \psi'_k = a_0 \psi_k.
\end{align}
We call the linear space spanned by the set $\{\psi_k\}_k$ as $H$ and the space spanned by $\{\psi'_k\}_k$ as $H'$.
If we write the transformation of $\psi'$ as
\begin{align}
  u \psi'_k = \sum_j \psi'_j \bar{\Delta}_{jk}(u),
\end{align}
$\bar{\Delta}$ satisfies
\begin{align}
  \bar{\Delta}_{jk} = \Delta^*_{jk} (a_0^{-1} u a_0).
\end{align}
This is because
\begin{align}
  a_0^{-1} u a_0 \psi_k 
  &= 
  a_0^{-1}
  \sum_j
  \Bigl(
    \psi'_j \bar{\Delta}_{jk}(u)  
  \Bigr)
  \nonumber
  \\&=
  \sum_j \psi_j \bar{\Delta}^*_{jk}(u)  
\end{align}
Note that $\bar{\Delta}$ is an irreducible representation of $G$ as well.

\subsubsection{The case that $\Delta$ and $\bar{\Delta}$ are equivalent}
We first investigate the case that $\Delta$ and $\bar{\Delta}$ are equivalent.
We assume that $\bar{\Delta}(u) = \Delta^*(a_0^{-1} u a_0) = U^{-1} \Delta(u) U$, and introduce a new basis
\begin{align}
  \psi'_k = \sum_j \varphi'_j U_{jk}.
\end{align}
Then, $\psi$ and $\varphi$ satisfies
\begin{align}
  u \psi_k &= \psi_j \Delta_{jk} (u)
  \label{eq:upsi}
  \\
  u \varphi'_k &= \varphi'_j \Delta_{jk}(u)
  \label{eq:uvarphiprime}
  \\
  a_0 \psi_k &= \varphi'_j U_{jk}
  \label{eq:a0psi}
\end{align}
Eqs. (\ref{eq:upsi}) and (\ref{eq:uvarphiprime}) defines the mappings $u: H \to H$, $u: H' \to H'$ respectively.
Eq. (\ref{eq:a0psi}) defines the mapping $a_0: H\to H'$.

Here, we investigate the condition that the above Eqs. (\ref{eq:upsi})to (\ref{eq:a0psi}) defines a consistent and a unique corepresentation.
\begin{enumerate}
  \item Corepresentation of $a_0$: $a_0 \varphi'_k=??$\\
  We first investigate how $a_0$ operates to the elements in $H'$. We start from 
  \begin{align}
    a_0(a_0 \psi_k) = a_0^2 \psi. 
    \label{eq:a02_psi}
  \end{align}
  Calculating each side separately, we get\\
  \begin{align}
    \text{(LHS)} = a_0 (\varphi'_j U_{jk}) = (a_0 \varphi') U^*
  \end{align}
  \begin{align}
    \text{(RHS)} = a_0 (\varphi'_j U_{jk}) = \psi \Delta(a_0^2)
  \end{align}
  Thus,
  \begin{align}
    a_0 \varphi'_k = \psi_j (\Delta (a_0^2) (U^{-1})^*)_{jk} = \psi_j (\Delta (a_0^2) U^{T})_{jk}
    \label{eq:a0varphi_prime}
  \end{align}
  In other words, Eq. (\ref{eq:a0varphi_prime}) specifies the mapping $a_0: H' \to H$ that satisfy Eq. (\ref{eq:a02_psi}).
  \item consistency of the transformation in $H$ and $H'$\\
  We start from the equation 
  \begin{align}
    (a_0^{-1} u a_0 ) \psi = a_0^{-1}(u (a_0 \psi)).
    \label{eq:consistency_transf_HHprime}
  \end{align}
  \begin{align}
    \text{(LHS)} = \psi_j \Delta_{jk}(a_0^{-1} u a_0).
  \end{align}
  To calculate the RHS, we use 
  \begin{align}
    a_0^{-1} \varphi'_k = \psi_j (U^{-1})^*_{jk} = \psi_j (U^{T})_{jk}, 
    \label{eq:inversea0_varphi}
  \end{align}
  which is obtained from $a_0^{-1} (a_0 \psi_k) = \psi_k$.
  \begin{align}
    \text{(RHS)} &= 
    a_0^{-1} u (\varphi'_j U_{jk}) 
    \nonumber
    \\&
    = a_0^{-1} (\varphi'_j (\Delta(u)U)_{jk})
    \nonumber
    \\&=
    \psi_j [(U^{-1})^* \Delta(u)^* U^*]_{jk}.
  \end{align}
  Therefore, we get
  \begin{align}
    \Delta(a_0^{-1} u a_0) = (U^{-1})^* \Delta(u)^* U^*.
    \label{eq:Delta_am1ua}
  \end{align}
\end{enumerate}

Now, we prove that the above two conditions [Eq. (\ref{eq:a0varphi_prime}) and (\ref{eq:Delta_am1ua})] are sufficient for $\Delta$ and $U$ to define a consistent corepresentation with the following three steps.
\begin{enumerate}
  \item When $u a_0 = a_0 v$, $u (a_0 \psi) = a_0 (v \psi)$.\\
  Note that $v = a_0^{-1} u a_0$. If Eq. (\ref{eq:Delta_am1ua}) is satisfied, we have
  \begin{align}
    a_0 v \psi_k 
    &= 
    a_0 (a_0^{-1} u a_0)\psi_k
    \nonumber
    \\&=
    a_0 (a_0^{-1} (u (a_0\psi_k))).
  \end{align}
  Using Eq. (\ref{eq:inversea0_varphi}), the last line can be rewritten as $u (a_0\psi_k)$. Thus, we get $u (a_0 \psi) = a_0 (v \psi)$.
  \item $a_0^2 \varphi' = a_0 (a_0 \varphi')$.\\
  When Eq. (\ref{eq:a0varphi_prime}) is satisfied, we get 
  \begin{align}
    a_0^2 \psi = a_0 (a_0 \psi).
    \label{eq:a02psi}
  \end{align} 
  In addition, we get $a_0^2 \psi = a_0^{-1} (a_0^2 (a_0 \psi))$ by substituting $u=a_0$ to Eq (\ref{eq:consistency_transf_HHprime}). Because the mapping from $H'$ to $H$ by $a_0^{-1}$ is defined as the inverse of $a_0$ [Eq. (\ref{eq:inversea0_varphi})], we get 
  \begin{align}
    a_0(a_0^2 \psi) =  a_0^2 (a_0 \psi).
  \end{align}
  Using Eq. (\ref{eq:a02psi}), we get 
  \begin{align}
    a_0(a_0(a_0 \psi)) =  a_0^2 (a_0 \psi).
  \end{align}
  Because $a_0 \psi$ spans a basis in $H'$, we get $a_0(a_0 \varphi') =  a_0^2 \varphi'$.
  \item When $u a_0 = a_0 v$, $u (a_0 \varphi') = a_0 (v \varphi')$.\\
  As $a_0 \varphi'$ spans a basis in $H$, we get
  \begin{align}
    u (a_0 (a_0 \varphi'_k)) = a_0 (v(a_0 \varphi'_k))
  \end{align}
  from Eq. (\ref{eq:consistency_transf_HHprime}). Using Eq. (\ref{eq:a02psi}) to the LHS, we get $u a_0^2 \varphi'_k$. Multiplying $a_0^{-2}$ from the left, 
  \begin{align}
    a_0^{-2} (\text{LHS}) &= a_0^{-2} (u a_0^{-2} \varphi')
    \nonumber
    \\&=
    (a_0^{-2} u a_0^{-2}) \varphi'
    \nonumber
    \\&=
    (a_0^{-1} (a_0^{-1} u a_0^{-1}) a_0)\varphi'
    \nonumber
    \\&=
    (a_0^{-1} v a_0)\varphi'
  \end{align}
  On the other hand,
  \begin{align}
    a_0^{-2} (\text{RHS}) 
    &=
    a_0^{-2} (a_0 (v(a_0 \varphi')))
    \nonumber
    \\&=
    a_0 (a_0^{-2} (v(a_0 \varphi')))
    \nonumber
    \\&=
    a_0 (a_0^{-1} (a_0^{-1} (v(a_0 \varphi'))))
    \\&=
    a_0^{-1} (v(a_0 \varphi'))
  \end{align}
  From the first to the second line, we used the special case of $u (a_0 \psi) = a_0 (v \psi)$ (Note that $v(a_0 \varphi')) \in H$).
  From the second to the third line, we use the inverse of Eq. (\ref{eq:a02_psi}). Note that we define the mapping $a_0^{-1}: H \to H'$ as the inverse of $a_0: H' \to H$.
  Thus, we get
  \begin{align}
    (a_0^{-1} v a_0)\varphi' = a_0^{-1} (v(a_0 \varphi'))
  \end{align}
  $u (a_0 \varphi') = a_0 (v \varphi')$ can be proven in a similar way as $u (a_0 \psi) = a_0 (v \psi)$
  \item For arbitrary $g_1, g_2 \in \bar{G}$, $(g_1 g_2) \psi = g_1 (g_2 \psi)$, $(g_1 g_2) \varphi' = g_1 (g_2 \varphi')$.\\
  It can be proven by combining the above lemmas.


\end{enumerate}







\bibliographystyle{unsrt}
\bibliography{reference}
\end{document}