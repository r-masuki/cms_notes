\documentclass{article}
\usepackage[top=25truemm,bottom=20truemm,left=20truemm,right=20truemm]{geometry}

\usepackage{amsmath}
\usepackage{amssymb}
\usepackage{amsfonts}
\usepackage{mathrsfs}
\usepackage{latexsym}
\usepackage{bm}
\usepackage[dvipdfmx]{graphicx}
\usepackage{physics}
\usepackage{braket}
\usepackage{float}

\usepackage{comment}

\title{Notes on symmetry and group theory}
\author{Ryota Masuki}
\date{\today}

\begin{document}
\maketitle

\section{Time-reversal and the representations of non-unitary group}
This section is based on the discussion in the textbook by Inui, et al~\cite{inui2012group}.

\subsection{Time-reversal operation}
If we define the time-reversal operator as
\begin{align}
  \theta = - i \sigma_y K = 
  \begin{pmatrix}
     & -1 \\
    1 &  \\
  \end{pmatrix}
  K,
\end{align}
where $K$ is the complex conjugate operator.

\subsection{Non-unitary group and corepresentations}
When the unitary operations $u_i, u_j, u_k$ satisfy $u_i u_j = u_k$, 
\begin{align}
  (\theta u_i) u_j = u_i (\theta u_j) = \theta u_k
\end{align}
\begin{align}
  (\theta u_i) (\theta u_j) = (-1)^N u_k,
\end{align}
where $N$ is the number of electrons. The magnetic group $\bar{G}$ can be written as
\begin{align}
  \bar{G} = G + \theta G,
\end{align}
where $G$ is the set of unitary symmetry operations (symmetry operations without time-reversal).

The corepresentations of the non-unitary group satisfy
\begin{align}
  u_1 u_2 \psi_k &= \sum_j \psi_j [D(u_1) D(u_2)]_{jk}
  \label{eq:corepresentation_u1u2}
  \\
  u a \psi_k &= \sum_j \psi_j [D(u) D(a)]_{jk}
  \label{eq:corepresentation_ua}
  \\
  au \psi_k &= \sum_j \psi_j [D(a) D^*(u)]_{jk}
  \label{eq:corepresentation_au}
  \\
  a_1 a_2 \psi_k &= \sum_j \psi_j [D(a_1) D^*(a_2)]_{jk},
  \label{eq:corepresentation_a1a2}
\end{align}
where $u, u_1, u_2$ are unitary operations and $a, a_1, a_2$ are anti-unitary operations in the group (We use this notation without explanation from here on).
Two corepresentations $D$ and $D'$ are defined to be equivalent if there is a matrix $T$ that satisfy
\begin{align}
  D'(u) &= T^{-1} D(u) T
  \label{eq:equiv_corep1}
  \\
  D'(a) &= T^{-1} D(a) T^*
  \label{eq:equiv_corep2}
\end{align}
Note that the relation Eqs. (\ref{eq:corepresentation_u1u2}) to (\ref{eq:corepresentation_a1a2}) is unchanged by the transformation Eq. (\ref{eq:equiv_corep1}) and (\ref{eq:equiv_corep2}).

\subsection{Construction of the corepresentations}
Here, we consider how to construct the corepresentations of $\bar{G}$ from the representations of its unitary part $G$.
We first start from an irreducible representation and a basis of $G$
\begin{align}
  u \psi_k = \sum_{j = 1}^d \Delta_{jk}(u).
\end{align}
Here, we define 
\begin{align}
  \psi'_k = a_0 \psi_k
\end{align}
If we write the transformation of $\psi'$ as
\begin{align}
  u \psi'_k = \sum_j \psi'_j \bar{\Delta}_{jk}(u),
\end{align}
$\bar{\Delta}$ satisfies
\begin{align}
  \bar{\Delta}_{jk} = \Delta^*_{jk} (a_0^{-1} u a_0).
\end{align}
This is because
\begin{align}
  a_0^{-1} u a_0 \psi_k 
  &= 
  a_0^{-1}
  \sum_j
  \Bigl(
    \psi'_j \bar{\Delta}_{jk}(u)  
  \Bigr)
  \nonumber
  \\&=
  \sum_j \psi_j \bar{\Delta}^*_{jk}(u)  
\end{align}
Note that $\bar{\Delta}$ is an irreducible representation of $G$ as well.

\subsubsection{The case that $\Delta$ and $\bar{\Delta}$ are equivalent}
We first investigate the case that $\Delta$ and $\bar{\Delta}$ are equivalent.
We assume that $\bar{\Delta}(u) = \Delta^*(a_0^{-1} u a_0) = U^{-1} \Delta(u) U$, and introduce a new basis
\begin{align}
  \psi'_k = \sum_j \varphi'_j U_{jk}.
\end{align}
Then, $\psi$ and $\varphi$ satisfies
\begin{align}
  u \psi_k &= \psi_j \Delta_{jk} (u)
  \nonumber
  \\
  u \varphi'_k &= \varphi'_j \Delta_{jk}(u)
  \nonumber
  \\
  a_0 \psi_k &= \varphi'_j U_{jk}
\end{align}







\bibliographystyle{unsrt}
\bibliography{reference}
\end{document}