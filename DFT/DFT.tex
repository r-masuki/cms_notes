\documentclass{article}

\usepackage{amsmath}
\usepackage{amssymb}
\usepackage{amsfonts}
\usepackage{mathrsfs}
\usepackage{latexsym}
\usepackage{bm}
\usepackage[dvipdfmx]{graphicx}
\usepackage{physics}
\usepackage{braket}
\usepackage{float}



\title{Notes on density functional theory}
\author{Ryota Masuki}
\date{\today}

\begin{document}
\maketitle

\section{Pseudopotential methods}
\subsection{Projector Augmented Wave (PAW) method}
\subsubsection{Theory}
The all electron wavefunction $\ket{\Psi}$ is written by 
\begin{align}
\ket{\Psi} = \mathcal{T} \ket{\widetilde{\Psi}}
\end{align}
using the pseudo wavefunction $\ket{\widetilde{\Psi}}$. Here, the PAW transformation, which transforms the PS wavefunction to AE wavefunction, is 
\begin{align}
  \mathcal{T} = 1 + \sum_{\bm{R}\alpha} \mathcal{T}_{\bm{R}\alpha},
\end{align}
where $\mathcal{T}_{\bm{R}\alpha}$ operates in a sphere surrounding the atom $\bm{R}\alpha$.

Here, for a set of complete set in the sphere $\ket{\widetilde{\phi}_{\bm{R}\alpha, nL}}$, we assume that 
\begin{align}
  \ket{{\phi}_{\bm{R}\alpha, nL}} = (1 + \mathcal{T}_{\bm{R}\alpha}) \ket{\widetilde{\phi}_{\bm{R}\alpha, nL}}.
\end{align}
Then, using the projecor functions $\ket{\widetilde{p}_{\bm{R}\alpha,nL}}$, which satisfies
\begin{align} 
  \sum_{nL} \ket{\widetilde{\phi}_{\bm{R}\alpha, nL}} \bra{\widetilde{p}_{\bm{R}\alpha, nL}} = 1
\end{align}
for all atoms $\bm{R}\alpha$, the PAW transformation can be explicitly written as 
\begin{align}
  \mathcal{T} = 1 + \sum_{i = (\bm{R}\alpha, nL)} (\ket{\phi_i} - \ket{\widetilde{\phi}_i}) \bra{\widetilde{p}_i}.
\end{align}

\subsubsection{Operators}
For any one-body operator $A$, 
\begin{align}
  \widetilde{A} 
  &= \mathcal{T}^\dag A \mathcal{T}
  \\&=
  A + 
  \sum_{i = (\bm{R}_1\alpha_1,n_1L_1),j = (\bm{R}_2\alpha_2, n_2 L_2)}
  \ket{\widetilde{p}_i}
  (\braket{\phi_i | A | \phi_j} - 
  \braket{\widetilde{\phi}_i | A | \widetilde{\phi}_j})
  \bra{\widetilde{p}_j}
  + \Delta A
\end{align}
Please see Ref.~\cite{PhysRevB.50.17953} for details of $\Delta A$.
When $A$ is a semilocal operator, $\Delta A$ vanishes, and the second term can be rewritten as 
\begin{align}
  \widetilde{A} 
  &= \mathcal{T}^\dag A \mathcal{T}
  \\&=
  A + 
  \sum_{\bm{R}\alpha,n_1L_1, n_2 L_2}
  \ket{\widetilde{p}_{\bm{R}\alpha,n_1L_1}}
  (\braket{\phi_{\bm{R}\alpha,n_1L_1} | A | \phi_{\bm{R}\alpha,n_2L_2}} - 
  \braket{\widetilde{\phi}_{\bm{R}\alpha,n_1L_1} | A | \widetilde{\phi}_{\bm{R}\alpha,n_2L_2}})
  \bra{\widetilde{p}_{\bm{R}\alpha,n_2L_2}}
\end{align}
The charge density is given by
\begin{align}
  n(\bm{r}) = \widetilde{n}(\bm{r}) + n^1(\bm{r}) - \widetilde{n}^1(\bm{r}),
\end{align}
where
\begin{align}
\widetilde{n}(\bm{r}) = \frac{1}{N} \sum_{n\bm{k}} f_{n\bm{k}} |\psi_{n\bm{k}}(\bm{r})|^2 
\end{align}
\begin{align}
  {n}^1(\bm{r}) = \frac{1}{N} \sum_{n\bm{k}} \sum_{\bm{R}\alpha, nL, n'L'} f_{n\bm{k}} 
  \braket{\psi_{n\bm{k}} | \widetilde{p}_{\bm{R}\alpha, nL}} 
  \phi^*_{\bm{R}\alpha, nL}(\bm{r})
  \phi^*_{\bm{R}\alpha, n'L'}(\bm{r})
  \braket{ \widetilde{p}_{\bm{R}\alpha, n'L'} | \psi_{n\bm{k}} } 
\end{align}
\begin{align}
  \widetilde{n}^1(\bm{r}) = \frac{1}{N} \sum_{n\bm{k}} \sum_{\bm{R}\alpha, nL, n'L'} f_{n\bm{k}} 
  \braket{\psi_{n\bm{k}} | \widetilde{p}_{\bm{R}\alpha, nL}} 
  \widetilde{\phi}^*_{\bm{R}\alpha, nL}(\bm{r})
  \widetilde{\phi}^*_{\bm{R}\alpha, n'L'}(\bm{r})
  \braket{ \widetilde{p}_{\bm{R}\alpha, n'L'} | \psi_{n\bm{k}} } 
\end{align}

\begin{figure}[H]
  \begin{center}
  \includegraphics[width=100mm]{figures/PAW_charge.png}
  \caption{Schematic of the PAW charge densities.}
  \end{center}
\end{figure}



\bibliographystyle{unsrt}
\bibliography{reference}
\end{document}